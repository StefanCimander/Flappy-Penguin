\documentclass{article}

\usepackage{graphicx}
\usepackage{subfig}

\begin{document}

\title{Flappy Penguin Projekttagebuch}
\author{Stefan Cimander}
\date{}

\maketitle


\section*{19. Juni 2016}

\begin{description}
  \item[Projektbeginn] Ein Teil der HCI-Vorlesung wurde von unserer Gruppe dazu genutzt, Ideen für das Abschlussprojekt zu generieren und daraus ein konkretes Konzept zu erarbeiten. Folgende Einschränkungen wurden dabei vorgegeben:
  \begin{itemize}
  	\item Im Rahmen des Projekts soll ein Spiel entwickelt werden
  	\item Dabei soll der Herzschlags- oder Atemsensor verwendet werden
  	\item Das System soll den Q-Learning-Algorithmus integrieren
  	\item Ein Alltagsgegenstand (z.B. Kuscheltier) soll als Steuergerät dienen.
  \end{itemize} 
\end{description}









	
%	This weeks tasks:
%    \begin{itemize}
%        \item Idea generation
%        \item Early drafts
%        \item Early game prototype
%    \end{itemize}
%
%        \subsubsection{Final Concept: Flappy Penguin (working name)}
%            A game in the spirit of \em Flappy Bird \em featuring:
%
%            \begin{itemize}
%                \item ``Protagonist'' is a penguin
%                \item Side-scrolling movement
%                \item Ice blocks entering the stage as obstacles
%                \item \em Physical object \em controls penguin movement
%                \item Breath-meter indicates penguins remaining breath
%                \item User breathing while the penguin is under water reduces the breath-meter
%                \item User breathing while the penguin is at the water surface replenishes breath-meter
%                \item Empty breath meter means death by suffocation, i.e. game over
%                \item Additional air bubbles can be collected under water to increase breath
%                \item \em Q-learning \em used for either placement of air bubbles of a second, computer-controlled penguin
%                \item Visually using the style of \em Thomas was Alone \em
%            \end{itemize}
%
%\clearpage
%
%    \subsection{Early drafts}
%        See Fig. \ref{fig:concepts}
%        \begin{figure}[h]
%            \subfloat[Mockup]{
%                \includegraphics[width=0.45\textwidth]{img/02_mockup.png}
%            }
%            \caption{Concept}
%            \label{fig:concepts}
%        \end{figure}
%
%\clearpage
%
%    \subsection{Early game prototype}
%        See Fig. \ref{fig:prototype_screenshots}
%        \begin{figure}[h]
%            % \subfloat[Early prototype]{
%                % \includegraphics[width=0.45\textwidth]{img/03_prototype_screenshot.png}
%            % }
%            \subfloat[Early game assets]{
%                \includegraphics[width=0.45\textwidth]{img/04_penguin_assets.png}
%            }
%            \subfloat[Prototype using first few assets]{
%                \includegraphics[width=0.45\textwidth]{img/05_prototype_screenshot.png}
%            }
%            \subfloat[Prototype using first few assets]{
%                \includegraphics[width=0.45\textwidth]{img/06_prototype_screenshot.png}
%            }
%            \caption{Screenshots of different iterations}
%            \label{fig:prototype_screenshots}
%        \end{figure}

\end{document}
